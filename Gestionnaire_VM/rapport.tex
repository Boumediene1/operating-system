\documentclass{article}

\usepackage[utf8]{inputenc}

\title{Travail pratique \#3 - IFT-2245}
\author{Boumediene Boukharouba
20032279\\
Alexandre Deneault
20044305}

\begin{document}

\maketitle

\section{Description du travail}

Le troisième travail pratique consistait à implanter une simulation d'un 
algorithme de gestion de mémoire virtuelle. Notre programme devait être 
capable d'adresser un espace virtuel de $ 2^{16}$ octets. Il doit faire la 
traduction de l'adresse logique vers l'adresse physique en utilisant un TLB et 
une table des pages. Lorsque la page demandée ne se trouve pas en mémoire, il 
libère l'espace mémoire dans lequel il va charger la nouvelle page en évinçant 
la page qui s'y trouve, la sauvegardant dans le fichier lorsque nécessaire. Le 
programme effectue ensuite les opérations démandées (lecture et écriture). 
Lorsque le programme termine, il sauvegarde les pages qui ont été modifiées 
sur le disque/fichier.


\section{Recherche}

\subsection{TLB et table des pages}

Afin de réaliser le travail, nous avons revus les notions sur le TLB et la 
table des pages dans les notes de cours. Nous avons étudiés plus en détails la 
structure de chacun de ces composantes afin de pouvoir les traiter 
correctement. Nous avons aussi étendu nos recherches à la conversion des 
adresses logiques en adresses physiques. Revoir les notes du cours nous a 
semblé suffisant pour accomplir le travail.

\subsection{Algorithme de remplacement}

Nous avons aussi revus les différents algorithmes de remplacement de page. 
Nous avons finalement décider d'utiliser un algorithme FIFO pour le TLB et un 
mélange de FIFO et LRU de style "Clock" pour la table des pages. Ces deux 
algorithmes nous semblaient être adapté à la tâche et un bon compromis pour 
l'efficacité. 


\section{Implémentation}

\subsection{Mémoire physique}

Nous avons débuté le travail par l'implémentation du transfert des caractères 
du fichier vers la mémoire physique et de la mémoire physique vers le fichier, 
ainsi que la lecture et l'écriture des caractères en mémoire. Nous avons aussi 
ajouté une structure pour pouvoir enregistrer les pages qui se trouvent en 
mémoire physique sur le fichier lorsque le programme se termine.

\subsection{Table des pages}

Il n'y avait rien de compliqué à faire pour la table des pages. Nous avions 
déjà étudié et bien compris la structure de la table. Nous avons simplement 
complété les fonctions présentes en sauvegardant les données passées en 
paramètre aux bons endroits dans le tableau représentant les pages ou en 
retournant les informations demandées.

\subsection{TLB}

Nous avons ensuite complété la section sur le TLB. Nous parcourons le tableau 
des entrées du TLB pour voir si la page recherché s'y trouve. Nous retournons 
la "frame" correspondante si nous trouvons la page, sauf si la page est marqué 
en lecture seulement. Pour l'ajout, nous vérifions d'abord si la "frame" à 
ajouter fait partit des "frames" qui sont marqués par le TLB afin d'attraper 
les modifications de l'état de "lecture seulement" et les pages qui en ont 
évincées une autre qui se trouvait dans le TLB. Si ce n'est pas le cas nous 
évinçons l'entrée la plus vieille du TLB pour pouvoir y placer la nouvelle 
page.

\subsection{VMM}

Lorsque le vmm reçoit une commande, il commence par traduire l'adresse logique 
en adresse physique. Les 8 premiers bits représentent l'offset et les 8 
derniers, le numéro de page. L'adresse physique de la page est d'abord 
recherché dans le TLB. Si le TLB ne donne pas de résultat, il regardons dans 
la table des pages. Si la table des pages retourne le numéro de "frame", vmm 
ajoute l'entrée dans le TLB et continue la conversion de l'adresse, sinon la 
procédure de "page fault" est enclenchée. Lors d'un page fault, vmm choisit 
d'abord dans quel "frame" il va placer la nouvelle page à l'aide de 
l'algorithme "clock". Si nécessaire, il va enregistrer la page qui s'y trouve 
dans le fichier et marquer l'entrée de la table des pages invalide avant d'y 
installer la nouvelle page. Vmm va ensuite ajouter l'entrée dans la table des 
pages et dans le TLB. Vmm va ensuite terminé la conversion de l'adresse 
logique en adresse physique et appeler la fonction appropriée (read ou write).

\subsection{Tests}

Nous avons décidé d'écrire un programme en python pour générer les fichiers de 
test. Pour le premier test, nous avons choisit 32 pages. Nous avons ensuite 
écrit le programme de façon à choisir aléatoirement parmis 12 page à la fois. 
Il choisit aussi aléatoirement le nombre d'accès consécutifs qu'il fait sur 
cette page (entre 1 et 5). Le programme choisit aussi aléatoirement l'index du 
caractère dans la page et s'il effectue une lecture ou une écriture. Lorsqu'il 
écrit il écrit le caractère 'a'. Nous avons décidé d'utiliser cette technique 
pour générer le test puisqu'il aurait été facile de contrôler la performance 
du TLB en choisissant quand faire des accès séquentiels et quand accéder à une 
autre page. Nous avons pensé que guider des accès aléatoire représenterait 
mieux l'utilisation de quelqu'un qui ne connait pas l'algorithme utilisé.\\\\
Pour le deuxième test, nous avons testé le remplacement des pages en 
commençant par réécrire toute la mémoire. La première page est rempli de 'a', 
la deuxième est rempli de 'b', etc. Ensuite, le fichier de test accède une 
fois en lecture à chacune des pages. Finalement, il accède une fois en lecture 
puis une fois en écriture à toute les pages, y écrivant le caractère de la 
page suivante à la position 100 ('b' pour la première page, 'c' pour la 
deuxième page, etc.). De cette manière, il est facile de vérifier l'état du 
fichier après le test.


\section{Conclusion}
En conclusion, ce travail était intéressant. Les consignes et le code fournit, 
la structure du travail en général, étaient beaucoup mieux que pour le 
deuxième travail pratique. Nous n'avons pas eu à faire de grosse recherche et 
la matière nécessaire à la réalisation du travail avait déjà été vue en 
classe. Il nous a permis d'approfondir nos connaissances de la mémoire 
virtuelle. C'était un bon travail.


\end{document}
