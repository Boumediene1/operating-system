\documentclass{article}

\usepackage[utf8]{inputenc}
\usepackage{amsfonts}            %For \leadsto
\usepackage{amsmath}             %For \text
\usepackage{fancybox}            %For \ovalbox

\title{Travail pratique \#3}
\author{IFT-2245}

\begin{document}

\maketitle

{\centering \ovalbox{\large ¡¡ Dû le 27 avril à 23h59 !!} \\}

\newcommand \mML {\ensuremath\mu\textsl{ML}}
\newcommand \kw [1] {\textsf{#1}}
\newcommand \id [1] {\textsl{#1}}
\newcommand \punc [1] {\kw{`#1'}}
\newcommand \str [1] {\texttt{"#1"}}
\newenvironment{outitemize}{
  \begin{itemize}
  \let \origitem \item \def \item {\origitem[]\hspace{-18pt}}
}{
  \end{itemize}
}
\newcommand \Align [2][t] {
  \begin{array}[#1]{@{}l}
    #2
  \end{array}}

\section{Survol}

Ce TP vise à vous familiariser avec les éléments de la mémoire virtuelles.
Les étapes de ce travail sont les suivantes:
\begin{enumerate}
\item Lire et comprendre cette donnée.  Cela prendra probablement une partie
  importante du temps total.
\item Lire, trouver, et comprendre les parties importantes du code fourni.
\item Compléter le code fourni.
\item Écrire un rapport.  Il doit décrire \textbf{votre} expérience pendant
  les points précédents: problèmes rencontrés, surprises, choix que vous
  avez dû faire, options que vous avez sciemment rejetées, etc...  Le
  rapport ne doit pas excéder 5 pages.
\end{enumerate}

Ce travail est à faire en groupes de 2 étudiants.  Le rapport, au format
\LaTeX\ exclusivement (compilable sur \texttt{frontal.iro}) et le code sont
à remettre par remise électronique avant la date indiquée.  Aucun retard ne
sera accepté.  Indiquer clairement le(s) nom(s) au début de chaque fichier.

Si un étudiant préfère travailler seul, libre à lui, mais l'évaluation de
son travail n'en tiendra pas compte.  Si un étudiant ne trouve pas de
partenaire, il doit me contacter au plus vite.  Des groupes de 3 ou plus
sont \textbf{exclus}.

\newpage
\section{Introduction}

Dans ce travail pratique, vous devrez implémenter en langage C un programme
qui simule un gestionnaire de mémoire virtuelle par pagination (paging).
Votre solution simulera des accès mémoire consécutifs en traduisant les
adresses logiques en adresses physiques de 16 bits dans un espace d'adresses
virtuelle (virtual address space) de taille $2^{16}$ = 65536 bytes.

Votre programme devra lire et exécuter une liste de commandes sur des adresses
logiques.  Pour y arriver il devra traduire chacune des adresses logiques à son
adresse physique correspondante en utilisant un TLB (Translation Look-aside
Buffer) et une table de pages (page table).

\section{Mise en place}

Le code fournis utilise un fichier \texttt{GNUmakefile} qui vous aidera lors
du développement.  Le projet dépend des programmes \texttt{bison} et
\texttt{flex} disponibles sur les environnement GNU/Linux, Cygwin et OSX.

Ce TP est basé sur un projet du livre de référence utilisé dans le cours, et
il vous aidera à mettre en pratique les sections 8.5 (paging), 9.2 (demand
paging) et 9.4 (page replacement).

Vous trouverez dans ces section tous les concepts nécessaires.  Pour vous
simplifier la tâche, on simulera une mémoire physique qui ne contient que
des caractères imprimables (ASCII).  C'est-à-dire, pour une mémoire physique
de 8KB (32 frames par 256 bytes), chacune de ses entrées contiennent un
caractère parmi les 95 caractères imprimables ASCII possibles.
Plus spécifiquement, le code fourni comprendsdéjà les structures de données
de base pour un gestionnaire de mémoire avec les paramètres suivants
(src/conf.h):

\begin{itemize}
\item 256 entrées dans la page table
\item Taille des pages et des frames de 256 bytes
\item 8 entrées dans le TLB
\item 32 frames
\item Mémoire physique de 32754 bytes
\end{itemize}

\section{Description du projet}

Puisqu'on a un espace de mémoire virtuelle de taille $2^{16}$ on utilisera
des adresses logiques de 16 bits qui encodent le numéro de page et le
décalage (offset).

Par exemple l'adresse logique 1081 représente la page 4 avec un décalage
(offset) de 57.

Votre programme devra lire de l'entrée standard (stdin) une liste de
commandes de lecture ou d'écriture.  Vous devrez décoder le numéro de page
et l'offset correspondant et ensuite traduire chaque adresse logique à son
adresse physique, en utilisant le TLB si possible (TLB-hit) ou la table de
page dans le cas d'un TLB-miss.

\section{Traitement de Page Faults}

Dans le cas d'un TLB-miss (page non trouvé dans le TLB), le page demandée
doit être recherchée dans la table de pages. Si elle est déjà présente dans
la table de pages, on obtient directement le frame correspondant. Dans le
cas contraire, un page-fault est produit.

Votre programme devra implémenter la pagination sur demande (section 9.2 du
livre). Lorsque un page-fault est produit, vous devez lire une page de
taille 256 bytes du fichier \texttt{BACKING\_STORE.txt} et le stocker dans un
frame disponible dans la mémoire physique (au début du programme la mémoire
physique commence toujours vide).

Par exemple, si l'adresse logique avec numéro de page 15 produit un
page-fault, votre programme doit lire la page 15 depuis le
\texttt{BACKING\_STORE.txt} (rapellez-vous que les pages commencent
à l'index 0 et qu'elles font 256 bytes) et copier son contenu dans une frame
libre dans la mémoire physique.  Une fois ce frame stocké (et que la table
de pages et le TLB sont mis à jour), les futurs accès à la page 15, seront
adressé soit par le TLB ou soit par la table de page jusqu'à ce que la page
soit déchargé de la mémoire (swapped out). Le fichier
\texttt{BACKING\_STORE.txt} est déjà ouvert et fermé pour vous. Il contient
65536 caractères imprimables aléatoires. Suggestion: utilisez les fonctions
de stdio.h pour simuler l'accès aléatoire à ce fichier.

\section{Commandes}

Les commandes sont automatiquement lues par les fichiers générés par
\texttt{flex} et \texttt{bison}.  Les fonctions \texttt{vmm\_read} et \texttt{vmm\_write} sont
automatiquement appelées.  Vous ne devriez donc vous préoccuper du
fonctionnement du programme qu'à partir de la gestion des commandes lues.

La lecture des commandes se fait par l'entrée standard (\texttt{stdin}) et les
commandes invalides sont ignorées. Les commandes sont insensible à la casse
et les espaces sont ignorés.  Les commandes sont de la forme suivante:

\begin{center}
  \begin{tabular}{ll}
    \begin{tabular}{l}
      commande d'écriture \\ \hline \hline
      \texttt{W} \id{logical-address} '\id{char-to-write}'; \\
      ex: \texttt{W20'b';}
    \end{tabular} &
        \begin{tabular}{l}
          commande de lecture \\ \hline \hline
          \texttt{R} \id{logical-address}; \\
          ex: \texttt{R89;}
        \end{tabular}
  \end{tabular}
\end{center}

\subsection{Makefile}

Pour vous faciliter le travail, le fichier \texttt{GNUmakefile} de l'archive
vous permet d'utiliser les commandes suivantes:
\begin{itemize}
\item \texttt{make} ou \texttt{make all}: Compile l'application.
\item \texttt{make release}: Archive le code dans un \kw{tar} pour la remise.
\item \texttt{make run}: Lance le client et le serveur ensembles.
\item \texttt{make clean}: Nettoie le dossier build.
\end{itemize}

\section{Travail à effectuer}

Vous devez implémenter les fonctions incomplètes de \texttt{vmm.c},
\texttt{pm.c}, \texttt{pt.c}, et \texttt{tlb.c}, y compris l'implémentation
de l'algorithme de remplacement du TLB et des frames ainsi que la gestion de
l'état ``dirty'' des pages.

De plus, vous devez corriger les sorties déjà définies dans les fonctions
\texttt{vmm\_read} et \texttt{vmm\_write} afin de fournir l'ensemble des
valeurs qu'il faut afficher.  

Finalement, vous devez fournir 2 fichiers de tests
\texttt{tests/command1.in} et \texttt{tests/command2.in}.  Le premier
devrait principalement tester l'efficacité du TLB alors que le deuxième
devrait aussi tester l'algorithme de remplacement des pages.

\section{Remise}

Remettez sur Studium l'archive générée par la commande \texttt{make release}.
Assurez-vous que tout fonctionne correctement sur \texttt{frontal.iro}.

\section{Détails}

\begin{itemize}
\item La note (sur un maximum de 15 points) sera divisée comme suit:
  3 points pour le rapport, 3 points pour les tests, 5 points sur le
  fonctionnement correct du code, 2 points sur la qualité du code, et
  2 points sur l'efficacité de vos algorithmes de remplacement (en terme de
  minimisation des TLB miss et des page faults).
\item Tout usage de matériel (code ou texte) emprunté à quelqu'un d'autre
  (trouvé sur le web, ...) doit être dûment mentionné, sans quoi cela sera
  considéré comme du plagiat.
\item Le code ne doit en aucun cas dépasser 80 colonnes.
\item Vérifiez la page web du cours, pour d'éventuels errata, et d'autres
  indications supplémentaires.
\item La note sera basée d'une part sur des tests automatiques, d'autre part
  sur la lecture du code, ainsi que sur le rapport.  Le critère le plus
  important, et que votre code doit se comporter de manière correcte.
  Ensuite, vient la qualité du code: plus c'est simple, mieux c'est.
  S'il y a beaucoup de commentaires, c'est généralement un symptôme que le
  code n'est pas clair; mais bien sûr, sans commentaires le code (même
  simple) est souvent incompréhensible.  L'efficacité de votre code est sans
  importance, sauf s'il utilise un algorithme vraiment particulièrement
  ridiculement inefficace.
\end{itemize}

\end{document}
