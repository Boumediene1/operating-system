\documentclass{article}

\usepackage[utf8]{inputenc}

\title{Travail pratique \#2 - IFT-2245}
\author{Boumediene Boukharouba
20032279\\
Alexandre Deneault
20044305}

\begin{document}

\maketitle

\section{Description du travail}

Le deuxième travail pratique consistait à implanter l'algorithme du banquier 
dans une structure client-serveur. Le programmes client et serveur devaient 
utiliser des socket pour communiquer. À l'intérieur des deux programmes, 
plusieurs threads client et serveur doivent s'exécuter en même temps en 
évitant les conditions de course dans les sections critiques. Les clients 
doivent demander des ressources au serveur qui doit utiliser l'algorithme du banquier 
afin d'éviter les interbloquages.


\section{Recherche}

\subsection{Multithreading}

Le multithreading est une notion fondamentale pour accomplir ce TP. De ce fait, nous
avons, dans un premier temps,révisé les notes de cours à ce sujet. Toutesfois, ce n'était 
pas suffisant. Par conséquent, nous avons cherché dans la documentation la sémantique exacte
des méthodes à utiliser. 


\subsection{Prévention des conditions de courses}

Pour la prévention des séquencement critiques, les notes de cours 
étaient suffisante, à l'exception de l'utilisation des sémaphores oû il a fallut vérifier 
la syntaxe des méthodes dans la documentation.


\subsection{Évitement d'interblocage}

L'algorithme du banquier a été bien expliqué en classe. Nous avons simplement revu 
l'algorithme dans les notes de cours afin de bien l'implémenter.

\subsection{Communication entre processus via socket}

La partie la plus longue du travail a été de faire la recherche sur les sockets 
et réussir à bien les utiliser. Cette matière n'a pas été vu en classe. Il n'y 
avait qu deux diapositives et deux illustrations dans les notes de cours à ce sujet.
Nous avons eu besoin de 2-3 jours pour faire la recherche et être en mesure de les 
utiliser. Nous avons d'abord cherché dans la documentation, cependant elle ne fait 
que décrire très brievèment ce que chaque méthode fait. Elle ne donne aucun indice 
sur quelle méthode utiliser et à quel endroit les utiliser (du côté client ou du 
côté serveur). Il a fallut chercher des tutoriels en ligne, mais la pluspart de 
ceux-ci n'expliquait pas bien le fonctionnement des méthodes et leur utilité. Nous 
avons fini par trouver un tutoriel qui expliquait l'utilisation des méthodes et des 
structures mieux que les autres (http://www.beej.us/guide/bgnet/output/html/singlepage/bgnet.html).
Il nous a permis de bien démarrer le travail.


\section{Implémentation}

\subsection{Communication client-serveur}

Nous avons débuté l'implémentation du code par l'utilisation des sockets puisque les 
données pour les autres parties du travail doivent voyager entre les deux processus. 
De plus, puisque nous n'avions pas vu cette notion en classe, nous savions que nous 
devions faire de la recherche pour bien comprendre ce que nous faisions. La recherche 
a pris beaucoup plus de temps que prévu! Après avoir bien compris l'utilisation des socket, 
nous avons créé une méthode pour ouvrir les sockets du côté client. Cette méthode ouvre 
la connexion avec le serveur et retourne le descripteur du fichier de la socket au thread 
client qui l'a appelé. Du côté serveur l'attente de la connexion d'un client était déjà 
implanté dans le code fournit. Par la suite, chacun des threads clients utilise le descripteur 
de fichier pour envoyer et recevoir des lignes de texte du serveur. 


\subsection{Traitement des requêtes}

\paragraph{}
Du côté serveur, nous avons créé une méthode afin d'extraire les paramètres des lignes 
de texte et retourner un tableau d'entier contenant la valeur des paramètres. Le 
programme client commence par envoyer les commandes "INI" et "PRO" au serveur à 
l'aide d'une première connexion. Le serveur met les paramètres de ces deux requêtes dans
des variables avant de répondre au client. Ensuite, le programme client démarre l'exécution 
des différents threads clients et se met en attente. Par la suite,les threads clients 
choisissent au hasard un nombre maximal de chacune des ressources qu'ils vont utiliser. 
Ils envoient ensuite la commande "INI" au serveur et attendent une réponse. Le serveur 
reçoit les commande et traite un client à la fois. Il enregistre les paramètres de la 
commande "INI" reçu du client dans le tableau de ressources maximales et initialise les 
ressources allouées à 0. Il répond ensuite au client. 

\paragraph{}
Suite à la réponse du serveur, le client commence à envoyer les requête. Il choisit au 
hasard le nombre de chaque ressource à demander au serveur(à l'exception de la dernière 
qui libère les ressources tenues). Il envoit alors la commande "REQ" au serveur. Lorsque 
le serveur accepte la requête, le client met à jour le tableau des ressources qu'il tient, 
met la variables qui compte les requêtes à jour  
et passe à la prochaine requête. À la réception de la requête du client, le serveur vérifie 
que la requête est valide et exécute l'algorithme du banquier. Si l'état est sûr, il accepte 
la requête. Si l'état n'est pas sûr, il attend un peu et recommence l'algorithme du banquier.
Le serveur met les variables qui comptent les requêtes reçu et accepté à jour avant de répondre 
au client.

\paragraph{}
Lorsque les clients ont terminé leurs requêtes, ils envoient la commande "CLO" au serveur 
afin de terminer leur exécution. Si les ressources ont bien été rendues, le serveur accepte 
la fermeture. Si ce n'est pas le cas, le serveur force le client à rendre les ressources avant 
de fermer la connexion. Le serveur et le client mettent les variables servant à compter les 
client à jour.

\paragraph{}
Chaque modification aux tableaux contenant les ressources maximales et les ressources allouées 
ainsi que les compteurs de requête est précédé d'un "wait" du sémaphore et suivi d'un "post". 
Nous avons décidé d'utiliser un seul sémaphore pour gérer les accès aux sections critiques par 
souci de simplicité. De plus, comme chaque section critique est très courte, nous ne penssons 
pas que l'utilisation de plusieurs sémaphores permettrait d'améliorer la performance.

\paragraph{}
Lorsque tout les clients ont terminé d'envoyer leurs requêtes et qu'il se sont fermé, le programme 
client envoit la commande "END" au serveur afin de le fermer. S'il ne reste plus de client connecté, 
le serveur se ferme et répond au client, lui signifiant qu'il peut se terminer à son tour. S'il reste 
des clients connecté au serveur lors du "END", le serveur force la fermeture des clients avant de se 
fermer. Les programmes client et serveur impriment leur rapport avant de terminer leur exécution.


\section{Conclusion}
En conclusion, ce travil a demandé énormément de recherche avant de pouvoir débuter la conception. 
De plus, recevoir le travil pendant les semaines d'examen nous a privé de beaucoup de temps pour 
faire ledit travil. Nous n'avons pas pu y toucher lors des deux premières semaines à cause de l'étude 
pour les examens. Cela dit, ce travail nous a permis de tester nos capacité d'autoapprentissage. 
Nous sommes assez fier du travail accomplit.





























\end{document}
